\documentclass[a4paper]{article}

%\usepackage[ngerman]{babel}								%Sprache
\usepackage[ansinew]{inputenc}						%Umlaute ohne Maske
\usepackage{amsmath}											%Mathe f�r alle!
\usepackage{amsfonts}											%Mathefonts
%\usepackage{graphicx}											%Bessere Grafikunterst�zung
%\usepackage[pdfborder = 0 0 0]{hyperref}	%Nette PDFs ohne M�h...


\begin{document}
	%%%%%%%%%%%%%%%%%%%%%%%%%%%%%%%%%%%%%%%%%%%%%%%%%%%%%%%%%
	%%% Hier Text einf�gen...
	
	\section{Handle devices}
	\subsection{Load a device}
	To load a device you can select "Load Device" in the main-menu "Load". 
	Or you can right-click on the devicelist and select the menu entry "Add Device".
	\par
	Now you'll see the deviceselection dialog. In this dialog all known devices are
	listed by there classes. You can open a class by double clicking the class.
	Now select the device you want to load and confirm it by clicking on the OK button. 
	\par
	Now you'll see the devicesetup dialog, where you can set all parameters needed to
	load and configure the device correctly. Fill up the form and click on the OK 
	button to confirm your settings. Note that you need to set at least the alias
	for the device, to indentify it later. If you need help for a specific
	parameter, keep a while with your mouse over the entryfield to get a
	tool-tip help. 
	\par
	If all going right, you'll see your device now in the devicelist. If an error 
	occours you can check out the logging window at the	bottom of the PPLT Center 
	window for the reason.  
		
		
	\subsection{Unload a device}
	To unload a device simply right-click on the device at the devicelist. Now
	select "Del Device" in the context menu. If the device can be unloaded the
	system will do so. Otherwise you will see a error message in the 
	loggingwindow at the bottom of the application. 
	\par
	\textbf{Note}: No symbols should be connected to the device if you want to
	unload the device.
	
	
	\section{Handle symbols and folders}
	\subsection{Create a folder}
	To create a folder please right-click at the folder where do you want to
	create the new folder. If you want to create a folder in the root directory
	simply right click at a free space in the symboltree or at the 
	\textit{Symbol Tree} tab. Then select \textit{Add Folder} in the contextmenu.
	Now you will be asked for a name of the folder. In this dialog you can also
	set the owner, gorup and access-rights for this folder. 
	\par
	\textbf{Note}: The PPLT Center
	allways sets this ownership by default to the superuser.


	\subsection{Rename a folder}
	To rename a folder, right-click the folder and select 
	\textit{rename} in the context-menu. Now the label 
	of the folder should become editable. Rename the folder
	and confirm by pressing RETURN. If the new name alredy 
	exists, an error message will be shown in the logging window
	at the bottom of the application and the old name will be 
	restored.
	
	
	\subsection{Move a folder}
	You can move a folder into an other or at the root by
	Drag'n'Drop.
	\par
	\textbf{Note}: You can't move a folder into a sub folder
	of the one, you want to move.


	\subsection{Delete a folder}
	To delete a folder right-click the folder and select 
	\textit{Del folder} in the context-menu. If the folder
	was empty it should be removed now. 
	\par
	\textbf{Note}: The folder but be empty.
	
	
	\subsection{Create a symbol}
	To create a symbol you should right-click the folder where you want to
	create the symbol. Or, if you want to create the symbol at the root
	directory, right-click at a free space inside the \textit{Symbol Tree}.
	Then select \textit{Add symbol} in the context menu. 
	\par
	Now you should see a dialog with a list of all loaded devices. Open the
	device, you want the symbol to be connected to, by a double-click. Now you should
	see a list of all namespaces, the device offers. Mostly, there are only one
	or two namespaces per device. Inside a namespace, you will find
	the slots of the device. Open the namespace by a double-click.
	\par
	Now select the slot you want to use by a double-click. You can see a
	short description of the slot at the bottom of the dialog , if you 
	single click the specific slot.
	\par
	If you selected a slot-range you will now be asked for a specific 
	address or name for the slot. This means that the application can't
	list for example all makers of a PLC, so the slot-rang is a placeholder
	for all markers and you have to specify now the maker you want to use.
	\par
	Now the symbol-property dialog pops up. Here you set the name and the
	ownership of the symbol. If you selected a slot-range you have to set also
	the type of the symbol. Also you can set here the chashing time for
	the symbol (in seconds).
	\par
	Click on the OK button to complete the setup and create the symbol.

	
	\subsection{Rename a symbol}
	To rename a symbol you can simply double-click the label of the
	symbol or right-click the symbol and select \textit{rename} at
	the context-menu. Now the label of the symbol should become 
	editable. Make your changes and confirm them by pressing RETURN.
	\par
	If the symbol can't be renamed, the old name will be restored and 
	a error message will be shown in the logging window at the bottom 
	of the application.
		
	\subsection{Move a symbol}
	Like a folder you can also move a symbol into an other
	folder or at the root-directory by drag and drop.

	\subsection{Delete a symbol}
	Deleteing a symbol is quiet the same like deleting a folder. Just right
	click the symbol you want to delete and select \textit{Del symbol} in
	the context menu. The symbol should now be removed.
	
	\subsection{Get or set values}
	To get or set the value of a symbol, you should right-click the symbol and
	select \textit{Get/Set value}. Now you should see a small dialog. You 
	can set the value of the symbol by changeing the entry \textit{Value}
	and then click the \textit{Set} button. 

	\subsection{Change owner, group and rights}
	To change the owner, group and the access rights of a symbol or folder right-click
	the item and select \textit{Properties} in the context-menu. Now you should
	see a dialog where you can set all these things. Confirm you changes by clicking
	the OK button.
	
	
	
	
	
	\section{Handle servers}
	\subsection{Load a server}
	Loading a server is quiet the same like loading a device. To do so, 
	select \textit{Load server} in the \textit{Load} menu or right-click
	at the server list and select \textit{Add server} in the context menu. 
	\par
	Now you should see all known servers listed by there classes. Open a class 
	by double click the classname. Now select the server you want to load
	by double click it. You can get a short information about  the
	server at the bottom of the dialog if you single-click the server.
	\par
	Now there should popup the setup-dialog of the server. Fill up the
	form and press the OK button to confirm. 

	\subsection{Unload a server}
	To stop and unload a server, please right-click the server you want to stop.
	Now select \textit{Stop server} in the context-menu. The server shoud now
	be unloaded and removed from the server list.
	
	
	\section{Userdatabase}
	\subsection{Create a user}
	To create a user right-click the group, the new user should be member of
	and select \textit{Add user} in the context-meun.
	\par
	Now you should see a dialog where you can set the user-name and also
	his password. Fill up the form and confirm you settings by clicking the
	OK Button.
	\par
	\textbf{Note}: A user must be a member of a group!
	
	\subsection{Change password of a user}
	To change the password of a user, right-click the user and select 
	\textit{Change password} in the context menu. Now there should popup
	a dialog where you can set the new password. Confirm your changes by
	clicking the OK button.
	
	\subsection{Delete a user}
	To delete a user, right-click the user and select \textit{Del user}
	in the context menu. The user should now be removed.
	
	\subsection{Create a group}
	If you want to create a subgroup right-click on the group where you want
	to create the subgroup. Else if you want to create a (root) group right-click
	on a empty field on the user-tree or on the tab \textit{UserDB}.
	Select \textit{Add group} in the context menu. Now you should see a 
	dialog where you can set the name of the new group. Click OK to finish.
	
	\subsection{Delete a group}
	To delete a group, right-click the group and select \textit{Del group}
	in the context-menu. Now the gorup should be removed.
	\par
	\textbf{Note}: The group must be empty (no users or subgroups in it) to
	delete it.
	
	\subsection{Create a user-proxy}
	To create a proxy for an user, right-click the group where you want to
	add the proxy. Select \textit{Add user-proxy} in the context-menu.
	Now you should see a dialog where you can select the user. 
	Click OK to finish.
	
	\subsection{Delete a user-proxy}
	To delete a user-proxy right-click the proxy and select 
	\textit{Del user-proxy} in the context-meun. Now the 
	proxy should be removed.
	
	\subsection{Set the superuser}
	Right-click the user, who should become the new superuser
	and select \textit{Set superuser}. This user should be now
	the superuser.
	
\end{document}