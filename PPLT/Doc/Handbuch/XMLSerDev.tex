Dieses Kapitel beschreibt den Syntax der Ger�te- und Serverdateien. Sie bilden 
den wichtigsten Teil der PPLT Abstraktionsschicht. Diese Beschreibungsdateien
im XML Format enthalten die Informationen, wie das System vorhandene Kernmodule
miteinander kombinieren muss, um ein Bestimmtes (reales) Ger�t anzusprechen oder 
einen bestimmten Dienst\footnote{Server} zur Verf�gung stellen kann.

Die Server und Ger�tedateien besitzen immer den selben Aufbau. Hier am Beispiel einer
Ger�tebeschreibungsdatei.

\begin{verbatim}
<?xml version="1.0"?>

<PPLTDevice name="Steuerung">
	<Head>
		<Description>...</Description>
		<Require>...</Require>
		<Provide>...</Provide>
	</Head>
	
	<Setup>
		...
	</Setup>
</PPLTDevice>
\end{verbatim}

Alle f�r das Ger�t relevanten Informationen sind in den \texttt{PPLTDevice}\footnote{Beim Server
hei�en diese \texttt{PPLTServer}.} Klammern eingeschlossen. Die Informationen werden in einen
Kopf (\texttt{Head}) und einen Setupbereich aufgeteielt. Im Kopf befinden sich alle beschreibenden 
Informationen wie einem kurzen Text, der das Ger�t und dessen Funktionen 
erleutert\footnote{\texttt{Description}}. Einem Block, der die Voraussetzungen, die dieses Ger�t 
ben�tigt\footnote{\texttt{Require}} beschreibt sowie einem Block indem alle Anschl�sse\footnote{Slots},
die das Ger�t beitet\footnote{Provide} auflistet.

Ich werde im folgenden den Aufbau der Beschreibungsdatei recht technisch aufschl�sseln. Es ist daher
eine Gewisse Vorkenntnis der PPLT von N�ten um die Ausf�hrungen verstehen zu k�nnen.

\section{\texttt{PPLTDevice} \texttt{PPLTServer}}
Innerhalb dieser Tags werden alle relevanten Informationen, die zur Beschreibung der Servers oder
des Ger�tes dienen, aufgelistet. Dieses Tag besitzt ein notwendige Attribute. Die Angabe dieser 
Attribute ist Pflicht.

\begin{tabular}{c|l}
\texttt{name}		& Gibt den Namen des Ger�tes oder des Servers an.\\
\texttt{class}	& Gibt den Namen der Klasse an, der dieses Ger�t oder Server zugeordnet werden soll.\\
\texttt{version}&	Gibt die Versionsnummer des Ger�tes an. 
\end{tabular}



\section{\texttt{Head}}
	Im Head werden wie Anfangs schon erw�hnt beschreibene Informationen zum Ger�t oder 
	Server aufgelistet. Immerhab der \texttt{Head}-Klammern d�rfen lediglich Elemente
	vom Namen \texttt{Description}, \texttt{Require} oder \texttt{Provide} stehen.
	
	Das Tag \texttt{Head} ben�tigt keinerlei Attribute.
	
\section{\texttt{Description}}
	Das Tag \texttt{Description} kann innerhalb vieler Tags verwendet werden, es enth�lt immer
	eine kurze Beschreibung des Kontextes, indem es steht. 
	
	Das \texttt{Description}-Tag kann innerhalb der Klammern \texttt{Head} und \texttt{Variable}
	stehen. Wenn das Tag im Kontext \texttt{Head} verwendet wird, kann damit dem gesamten Ger�t
	ein beschreibender Text gegeben werden. Wird das Tag im \texttt{Variable}-Kontext verwendet
	wird einer bestimmten Variable eine Beschreibung zugeordnet.
	
	Als Attribut muss dem \texttt{Description}-Tag immer \texttt{lang} �bergeben werden. Damit
	wird die Sprache des Beschreibungstextes fest gelegt. 	
	
\section{\texttt{Require}}
	Innerhalb der dieser Klammern, werden alle Voraussetzungen, die das Ger�t oder der Server
	ben�tigt aufgelistet.	
\section{\texttt{Setup}}

