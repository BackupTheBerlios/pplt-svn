\documentclass[a4paper]{book}

\usepackage[ngerman]{babel}								%Sprache
\usepackage[ansinew]{inputenc}						%Umlaute ohne Maske
%\usepackage{amsmath}											%Mathe f�r alle!
%\usepackage{amsfonts}											%Mathefonts
\usepackage{graphicx}											%Bessere Grafikunterst�zung
\usepackage[pdfborder = 0 0 0]{hyperref}	%Nette PDFs ohne M�h...


\title{PPLT-Handbuch}
\author{Hannes Matuschek}
\date{2005-05-25}


\begin{document}
	\begin{titlepage}
		\maketitle
		\vfill
		\tableofcontents
	\end{titlepage}
	\parindent = 0em
	\parskip = 2ex

\chapter{Einf�hrung}
	\section{Was ist PPLT?}
PPLT steht f�r \textit{Potsdamer Prozessleit-Technik} in Anlehnung an die 
\item{Achener Prozessleittechnik}\footnote{...acplt url...}. Die PPLT ist 
ein so genanntes Framework, f�r die Kommunikation mit Industriesteuerungen,
Sensoren, Akteuren oder mit Hausautomatesierungstechnik.

Das System ist quellofen und ist unter der GNU-GPL und der GNU-LGPL lizensiert.
Die GPL\footnote{General Public License} sichert jedem Benutzter, das die Quellen
des Programmes ihm ohne Einschrenkungen zug�nglich gemacht werden. Wenn jedoch
das Programm oder Teile davon ver�ndert oder in anderen Projekten verwendet werden,
m�ssen diese wiederum unter der GPL ver�ffentlicht werden. Um dennoch die M�glichkeit
zu bieten, dass die PPLT in anderen Projekten verwendet werden darf, sind die 
Bibliotheken, die den Gro�teil von PPLT ausmachen, unter der LGPL lizensiert. Dies 
bietet die M�glichkeit Teile der PPLT auch in eigenen Projekten zu verwenden und diese
dann unter einer bliebigen Lizenz\footnote{durchaus auch properit�re Lizenzen} zu
ver�ffentlichen. 

Geschrieben wurde die PPLT mit der Programmiersprache Python. Diese ist �hnlich
wie Java eine Umgebung mit der platfomunabh�nige Programme geschrieben werden
k�nnen. Somit ist auch die PPLT ein Platform unabh�niges System. 


\begin{itemize}
	\item Potsdamer Prozessleit-Technik
	\item quelloffen, (GPL,LGPL)(mit Erkl�rung)
	\item Python (wiki)
	\item platformunabh�ngig
\end{itemize}

	\subsection{Ziele}
	\begin{itemize}
		\item Ziele
		\item Automatisierung
	\end{itemize}
	
	\subsection{Grobe Funktionsweise}
	\begin{itemize}
		\item modular (module, plugins)
		\item plugins erkl�ren (wiki)
	\end{itemize}

	\section{Technische Details}
\begin{itemize}
	\item PPLT vs pyDCPU
	\item Schalenmodell
	\item wozu es benutzt werden kann
	\item Grenzen (Nur Master im BUS, kein Multimaster-Betrieb, Polling)
	\item Vorteile: module schachtelbar, eigene Benutzerverwaltung, nicht beschr�nkt auf
	eine applikation (visualisierung)
	\item Modularisierung (Kern-Module, Ger�te, Server)
\end{itemize}	

\section{Was ben�tige ich?}



\chapter{Erste Schritte}
\section{PPLT Installieren}
\subsection{PPLT Center}



\end{document}
