\newcommand{\PSFTag}[1]{\index{PSF Elements!#1}}



\chapter{PSF Reference --- \textbf{P}PLT \textbf{S}ession \textbf{F}ile}

This section describes the file-format of the PPLT session files. Such 
files contain the information about the state of the system. 
So the file contains information about the loaded devices and servers and
also the layout of the symbol-tree.

Files formatted in the PSF format can be created from a running PPLT system by 
a call of the method \method{SaveSession()} and can then be loaded by a 
\method{LoadSession()} method-call.

By this way you can easily implement a simple daemon or service, that reads 
such a PSF file and setup a whole PPLT system. Meaning loading all devices, 
creating all symbols and folders and at the end loading all servers.

All you need to know is a basic knowledge of XML formats, and a little idea of 
what PPLT is and what the system needs to save (or even needs to know to setup
the PPLT system). 

\section{Reference}
In the following section I'll list all keyword of the PSF format. These are only few
because the format stills easy. 

At first: The PSF format is an XML format!

\subsection{\keyword{PPLTSession}}
\PSFTag{PPLTSession}
The PPLTSession tags surround the PPLT session document. Containing all 
information about the devices, servers and symbol-tree. This tag must
(directly) contain the elements (tags) \keyword{Servers}, \keyword{Devices}
and \keyword{SymbolTree}. But each elements can only occur once.
This also defines the basic layout of the PPLT session document. Each document
should look like:
\begin{verbatim}
<?xml version="1.0"?>
<PPLTSession>
    <Servers>
        [...]
    </Servers>
    <Devices>
        [...]
    </Devices>
    <SymbolTree>
        [...]
    </SymbolTree>
</PPLTSession>
\end{verbatim}


\subsection{\keyword{Servers}}
\PSFTag{Servers}
The tag \keyword{Servers} contains the description of all loaded servers. This tag can only 
contain any occurrence of the tag \keyword{Server}. Each \keyword{Server}-tag describe a 
singe server to load. For more details see at the description of the \keyword{Server}-tag.

\begin{verbatim}
[...]
    <Servers>
        <Server [...]>
            [...]
        </Server>
        [...]
    </Servers>
[...]
\end{verbatim}


\subsection{\keyword{Server}}
\PSFTag{Server}
The tag \keyword{Server} describes a server to load. The tag can only contain any occurrence of the tag
\keyword{Parameter}. Each \keyword{Parameter} tag specifies a parameter needed to load the server.
Also the \keyword{Server} needs some attributes. The attribute \var{alias} specifies the alias the
server will get after it was loaded. The attribute \var{fqsn} specifies the full-qualified-server-name. 
This unique name specifies the server to be loaded. The Attribute \var{root} specifies the server root.
\note{The given server-root but exist else the loading of the server will fail.} The attribute 
\var{user} specifies the default user. The server will run with the rights of this user, if the 
server doesn't know any authentication. 

A simple example will can be:
\begin{verbatim}
[...]
    <Server alias="JVisu" fqsn="Visu.JVisuServer" root="/" user="admin">
        <Parameter name="Port">2200</Parameter>
        <Parameter name="Address">127.0.0.1</Parameter>
    </Server>
[...]
\end{verbatim}
This example will start a JVisu server with the rights of the user \code{admin} and
with the server root \code{'/'}. JVisu is a open source visualization written in Java.
You can get it from \url{http://jvisu.sourceforge.net}.

\subsection{\keyword{Parameter}}
\PSFTag{Parameter}
The \keyword{Parameter}-tag specifies a parameter, the system will use to setup
a \keyword{Server} or \keyword{Device}. The \keyword{Parameter}-tag can only 
contain CDATA, meaning it can only contain a string. The content will be 
interpreted as the value of the parameter. The \keyword{Parameter}-tag
knows also a attribute. The attribute \var{name} specifies the name
of the parameter. To find out what parameters are needed to setup
a specific server, please consult the Device and Server reference.


\subsection{\keyword{Devices}}
\PSFTag{Devices}
Like the \keyword{Servers}-tag this tag contains the description of
all devices needed to be loaded. So this tag can only contain any
occurrence of the tag \keyword{Device}. For example:
\begin{verbatim}
[...]
    <Devices>
        <Device [...]>
            [...]
        </Device>
        [...]
    </Devices>
[...]
\end{verbatim}


\subsection{\keyword{Device}}
\PSFTag{Device}
Like the \keyword{Server}-tag this tag describes one device to load. Also it can only contain 
any number of \keyword{Parameter} tags to specify the the parameters the device may need to be
loaded. It also has some attributes. The attribute \var{alias} specifies the alias the device 
gets when it's was successfully loaded. The attribute \var{fqdn} specifies the full qualified 
device name. This is the name of the device you want to load.

For example, a \keyword{Devices}-section can be:
\begin{verbatim}
[...]
    <Devices>
        <Device alias="rand" fqdn="Debug.RandomGenerator"/>
        <Device alias="S7" fqdn="PLC.S7-200">
            <Parameter name="PCAddr">0</Parameter>
            <Parameter name="Port">1</Parameter>
            <Parameter name="S7Addr">2</Parameter>
        </Device>
    </Devices>
[...]
\end{verbatim}
This example will load a random-generator as \code{'rand'}  and the support for the Siemens SIMATIC S7-200 
as \code{'S7'}.


\subsection{\keyword{SymbolTree}}
\PSFTag{SymbolTree}
The \keyword{SymbolTree}-section describes the content of the symbol-tree. So this tag can only contain
any occurrence of \keyword{Symbol} and \keyword{Folder} tags. This tag don't take any attributes.

\subsubsection{How the symbol-tree is maped}
The symbol-tree works like a small filesystem, with folders, sub-folder and symbols. The root of this filesystem is the
symbol-tree itself. Now you need to know how to map the hierarchy of the symbol-tree to an XML file.

I think the best way to describe is to give an example. The following symbol-tree hierarchy
\begin{verbatim}
/
/FOLDER1/
/FOLDER1/FOLDER2/
/FOLDER1/FOLDER2/Symbol1
/FOLDER1/FOLDER2/Symbol2
/FOLDER1/Symbol3
/FOLDER3/
/FOLDER3/Symbol4
/Symbol5
\end{verbatim}
will be maped as something like:
\begin{verbatim}
<SymbolTree>
    <Folder name="FOLDER1">
        <Folder name="FOLDER2">
            <Symbol name="Symbol1"/>
            <Symbol name="Symbol2"/>
        </Folder>
        <Symbol name="Symbol3"/>
    </Folder>    
    <Folder name="FOLDER3">
        <Symbol name="Symbol4"/>
    </Folder>
    <Symbol name="Symbol5"/>
</SymbolTree>
\end{verbatim}
As you see; the \keyword{SymbolTree} can contain \keyword{Symbol}s and \keyword{Folder}s. And each \keyword{Folder} can
also contain \keyword{Symbol}s and \keyword{Folder}s. But only the \keyword{Symbol}s are empty, because they can't contain
anything. In this example I missed some additional attributes to avoid you to be confused. This attributes are described 
in detail in the following sections.

\subsection{\keyword{Symbol}}
\PSFTag{Symbol}
The \keyword{Symbol}-tag describes a symbol of the symbol-tree. Such a symbol is a kind of a variable that can hold values.
Each symbol is connected to a so called "symbol-slot"\footnote{A symbol-slot always belongs to a device, so if you want to know 
what slots are provided by a specific device, pleas look at the documentation of the device.}. So you need to specify the 
symbol-slot the symbol will be connected to by the attribute \var{slot}. The slot-name must be provided in the format 
\code{'DEVICE\_ALIAS::NAMESPACE::SLOT\_NAME'}. The \code{DEVICE\_ALIAS} specifies the device you want the symbol to be
connected to. It should be the string you give to the \keyword{Device}-attribute \var{alias}. The \code{NAMESPACE}s and
\code{SLOT\_NAME}s are provided by the devices. To find out what namespaces are available, look at the documentation of 
each device.

In this context you have to set the refresh-rate of the symbol. This value in seconds specifies the time
a value will be cached by the symbol for the next read(s).

Now your symbol can get and (maybe) set values. But a major feature of the PPLT is, that you can control the
access to the symbol by setting rights to it. To do so, you need to set a \var{owner}, \var{group} and
\var{modus} to the symbol (and also to the folder). So a complete example for a symbol that is connected to
a random-generator loaded as \code{'rand'} could be:
\begin{verbatim}
[...]
    <SymbolTree>
        <Symbol name="r_bool" slot="rand::Generator::Bool" refresh="0.5" 
                owner="admin" group="Admin" modus="600"/>
        [...]
    </SymbolTree>
[...]
\end{verbatim}
You see the attribute \var{modus} contains a 3-digit integer. This integer is calculated like the Linux modus integer. Pleas look
at the \manpage{chmod}{} man-page to get more information. But sofar the number \code{600} says that only the owner has the right to 
read/write to the symbol and the group and any other has no rights.


\subsection{\keyword{Folder}}
\PSFTag{Folder}
The \keyword{Folder}-tag describes a folder in the symbol-tree. The name of the folder will be specified by the
attribute \var{name}. Like the symbols a folder also owned by a user and attached to a group. So you also need 
to set the \var{owner}, \var{group} and \var{modus} attributes. 

\newpage
\section{A complete Example}
In the following section I'll show you a complete example of a PSF. The file describes two devices, a random generator and
a Siemens SIMATIC S7-200. The symbol-tree contains two folders one, named 'Rand', will contain some symbols connected
to the random-generator and the other, named 'S7' will contains the symbols connected to the S7. And a web-server will be loaded, 
that exports the whole symbol-tree. 

\verbatiminput{PSF.xml}


