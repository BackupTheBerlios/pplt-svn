\chapter{PPLT Reference}
\section{\module{PPLT} ---
        \textbf{P}otsdamer \textbf{P}rozess \textbf{L}eit \textbf{T}echnik}

\declaremodule{}{PPLT}
\moduleauthor{Hannes Matuschek}{hmatuschek@gmx.net}
\modulesynopsis{The PPLT is a framework for master/slave based communication.}

The \module{PPLT} builds an abstraction layer over the core library 
(\module{pyDCPU}). With the PPLT you can easily handle abstract devices and 
server instead of handle with several core modules. Like the core-library the 
PPLT has only one class called \code{System}. After instancing this class, all
work can be done by calling the methods of this object.

\section{Class description}
\begin{classdesc}{System}{\optional{BasePath}, \optional{CoreLogLevel}, 
\optional{PPLTLogLevel}, \optional{LogFile}, \optional{Syslog}, 
\optional{Lang}, \optional{AltLang}}
This is the all-in-one class. If this class will be instanced the complete 
PPLT system will be loaded. Meaning searching plug-ins (modules), loading 
user-database, starting core-system, etc...

The argument \var{BasePath} specifies the path of all PPLT related stuff. This 
is by default \code{sys.exec\_prefix+'/PPLT'}. Under Windows this is the folder 
where you've installed Python under \UNIX this should be some thing like 
\code{/usr/local/PPLT}. If you miss this argument, the default path will be 
used.

The argument \var{CoreLogLevel} specifies the log-level for the core library. 
This should be one of \code{off}, \code{fatal}, \code{error}, \code{warning}, 
\code{info}, \code{debug}.  If you select \code{debug} a lot of (not always 
useful) messages will be logged and \code{off} switches the logging off. 
By default the logging level \code{info} will be used. If you miss this 
argument, the default value will be used.

The argument \var{PPLTLogLevel} specifies the log-level of the PPLT library. 
This should be one of \code{off}, \code{fatal}, \code{error}, \code{warning}, 
\code{info}, \code{debug}. If you select \code{debug} a lot of (not always 
useful) messages will be logged and \code{off} switches the logging off. 
By default the logging level \code{info} will be used. If you miss this 
argument, the default value will be used.

The argument \var{LogFile} specifies the file,  where the logging messages are
written to. If you missed this argument, there were no messages written to any
file, all messages were send to screen (\code{stderr}). 

The boolean argument \var{SysLog} specifies if the messages should be send to
the local \code{syslog}-daemon. 
\begin{notice}
This argument overwrites the optional argument \var{LogFile}. If \var{SysLog}
it \code{True} no messages were written into any file nor send to screen!
\end{notice}
By default this argument if \code{False}. If you missed this argument, the 
default will be used.

The argument \var{Lang} specifies the primary language the system will use for
module description. This option do not change the language of the logging 
messages (they are not translated). This option is not important for normal
usage of the PPLT. By default \code{'en'} will be used.

The argument \var{AltLang} specifies the alternative language to be used
for module description. This should be always \code{'en'}. \textbf{Do not 
change until you know what you do.} 

If the primary language can not be found, the alternative will be used. 
\end{classdesc}




\section{Methods of \class{System}}
In this section I will describe the methods of the \class{System}-class. I 
will not list all available methods, because there are more then 60 of it. Some
of them are only useful if you want to write a GUI application like the
\emph{PPLT Center}. But you can get a short help over all methods by calling
\code{pydoc PPLT.System}. 


\subsection{Device methods}
\begin{methoddesc}[System]{LoadDevice}{DeviceName, Alias, Parameters}
This is one of the most important methods. With this method you can load and
setup a device. This method loads the device \var{DeviceName} as \var{Alias}. 
The alias will be used later to specify this device, for example if you want
to connect a symbol to this device or unload the device.

The argument \var{DeviceName} specifies the full qualified device name. All
devices (and servers) are grouped by classes. A full qualified device name
consists always of all classes and the name divided by a single dot. For
example \code{'PLC.Panasonic-FPX'}.

The argument \var{Alias} specifies the alias the device will get after being
loaded. By this alias you will identify the device later.

The attribute \var{Parameters} specifies the parameters the device needs to be
set up successfully. This should be a dict with key-value pairs. The key 
is the parameter name and the value should be the parameter-value. For example:
\code{\{'Port':'0', 'Address':'123'\}}.
\begin{notice}
All parameter names and also \textbf{all} parameter values are strings!
\end{notice}
\end{methoddesc}


\begin{methoddesc}[System]{UnLoadDevice}{Alias}
This method unload and destroy the given device. 

The argument \var{Alias} specifies the alias of the device you want to be
unloaded. This is the alias you've set at the \method{LoadDevice} 
method-call.
\end{methoddesc}


\begin{methoddesc}[System]{GetFQDeviceName}{Alias}
This method returns the full qualified device name of the device loaded as
\var{Alias}.

The attribute \var{Alias} specifies the alias of the loaded device. This is
the alias you've set at the \method{LoadDevice} method-call.

This method returns a string containing the full qualified device name.
\end{methoddesc}


\begin{methoddesc}[System]{GetDeviceParameters}{Alias}
This method will return the parameters the device was loaded with. This method
will return a dict on success.
\end{methoddesc}


\subsection{Server methods}
This section describes all methods to handle servers, like starting, stopping, ... 

\begin{methoddesc}[System]{LoadServer}{ServerName, Alias, DefaultUser, Parameters\optional{, Root}}
This method loads and setup a server. 

The argument \var{ServerName} specifies the full qualified server name. Full 
qualified means that all classes and the name are given, divided by a single 
dot (like: \code{Web.PPLTWebServer}).

The argument \var{Alias} specifies the alias the server will have after 
being loaded. By this alias you will identify the server later, for example 
to unload this server.

The argument \var{DefaultUser} specifies the the user, the server will run as. 
By this option you can assign rights to a server, that doesn't know any 
authentication. By default, a server should support a authentication but 
sometimes the used protocol doesn't (for example JVisu). For this kind of 
server this option is useful.

The argument \var{Parameters} specifies the parameters the server will be 
loaded with. This should be a dict with the parameter name as key (string)
and the parameter value as value (also a string!). If a server doesn't needs
any parameters, please set \var{Parameters} to \code{None} or to and empty 
dict (\code{\{\}}). 

The optional argument \var{Root} specifies the server-root of this server. 
By default (\var{Root}=\code{'/'}) the whole symbol-tree will be accessible
by this server. But if you want to export only a specific folder of the 
symbol-tree you can specify this folder with the \var{Root} argument. So
if \var{Root}=\code{'/test'} only the content of the folder \code{'/test'}
will be accessible by this server.
\end{methoddesc}


\begin{methoddesc}[System]{UnLoadServer}{Alias}
This method will stop and destroy a server loaded with \method{LoadServer}.

The attribute \var{Alias} specifies the server you want to stop. This is the 
alias you've defined on loading the server with \method{LoadServer}.
\end{methoddesc}


\begin{methoddesc}[System]{GetFQServerName}{Alias}
This method will return the full qualified server-name of the loaded server
by his alias. 

The argument \var{Alias} specifies the alias of the server. This is the alias
you have defined on loading by the \method{LoadServer} method.
\end{methoddesc}


\begin{methoddesc}[System]{GetServerParameters}{Alias}
This method will return the dict of parameters the server was loaded with. 
This is the parameter dict, you have defined on loading the server with 
\method{LoadServer}.

The argument \var{Alias} specifies the alias of the loaded server. This is
the alias you have defined on loading the server with \method{LoadServer}.

This method will return an empty dict even if you've loaded the server with no
parameters.
\end{methoddesc}


\begin{methoddesc}[System]{GetServerRoot}{Alias}
This method will return the server root of a loaded server. This is the 
server root you have defined on loading the server with \method{LoadServer}.

The server root is the base folder in the symbol-tree the server can access. 
Only folders and symbols laying under this folder are accessible by this 
server.

The argument \var{Alias} specifies the alias of the loaded server. This is the
alias you have defined on loading the server with \method{LoadServer}. 

This method will return a string on success.
\end{methoddesc}


\begin{methoddesc}[System]{GetServerDefaultUser}{Alias}
This method will return the default-user the server is running with. This is
the username you have defined on loading the server with \method{LoadServer}.

If specified, the server will run with the rights of the default user. This is
necessary, because there are some servers which don't know any authentication!

The argument \var{Alias} specifies the alias of the loaded server. This is the 
alias you have defined for this server on loading with \method{LoadServer} 
call.

This method will return a string on success.
\end{methoddesc}


\subsection{Symboltree methods}
In the following section I will describe all methods for handling the symbol-tree. 
Like creating folder and symbols, moving them, ...

All methods, that working with the access-rights of a symbol or folder
are using following scheme to encode the rights:

\begin{center}
\begin{tabular}{|c|cc||c|cc||c|cc|} \hline\hline
\bf{Own} & $r$&$\not r$&\bf{Grp}&$r$ & $\not r$ & \bf{Any} & $r$ & $\not r$\\\hline
$w$ & 6 & 2 & - & 6 & 2 & - & 6 & 2 \\
$\not w$ & 4 & 0 & - & 4 & 0 & - & 4 & 0 \\\hline\hline
\end{tabular}
\end{center}
The access-right is represented as a string of 3 numbers each number specifies 
a access right. The first number specifies the right of the owner of the 
symbol or folder. The second specifies the right of the group assigned to the 
symbol/folder.  The last number specifies the right of everyone who not 
belongs to the assigned group nor being the owner. To en/decode the 
representation you can use the table above.


\begin{methoddesc}[System]{CreateFolder}{Path\optional{, Modus}\optional{, Owner}\optional{, Group}}
This method will crate a new folder (at \var{Path}) at the symbol-tree. 
Optional you can set the rights of this folder by the arguments \var{Modus}, 
\var{Owner}, \var{Group}. 

The argument \var{Path} specifies the \textbf{complete} path of the folder, 
you want to create. You cant create folders relative to an other because the 
system don't know the relative.

The argument \var{Modus} specifies the modus the folder will have. By this 
argument you can set the access rights of this folder. The argument should be
a string with an octal integer in it. Meaning something like this: \code{'600'}.
Each number of the integer represents the encoded right. For the owner of the
folder (1st number), the group assigned to the folder (2nd number), and any other
(last number). For encoding you can use table above. 
By default the modus will be \code{'600'}. This denotes that only the owner 
of the folder have read/write access. (By default this will be the system admin.)

The argument \var{Owner} specifies the owner of the folder. This have to be 
existing user-name. 

The argument \var{Group} specifies the group, the folder will be assigned to. 
This have to be an existing group-name.

\begin{notice}
You can reset the modus, owner and group later by calling 
\method{ChangeModus}, \method{ChangeOwner} or \method{ChangeGroup}.
\end{notice}
\end{methoddesc}


\begin{methoddesc}[System]{DeleteFolder}{Path}
With this method you can delete an \textbf{empty} folder of the symbol-tree.

The argument \var{Path} specifies the \textbf{full} path to the folder you want to
delete. This is the path you've set on create the folder by 
\method{CreateFolder} method-call.
\end{methoddesc}


\begin{methoddesc}[System]{MoveFolder}{From, To}
This method moves a folder to an other or to the root of the symbol-tree.

The argument \var{From} specifies the \textbf{full} path of the folder you want
to move. The argument \var{To} specifies the \textbf{full} path of the destination 
folder you want the source folder to be moved to. Of cause, the destination folder
can't be a sub-folder of the source.
\end{methoddesc}


\begin{methoddesc}[System]{ListFolders}{Path}
This method returns the names of all folders in \var{Path}.

The argument \var{Path} specifies the \textbf{full} path of the parent-folder
you want to list it's sub-folders. If you want to list the root path, please 
set \var{Path}=\code{'/'}.

The method returns a list of strings on success. This method will return an 
empty list if the given folder doesn't contain any sub-folders.
\end{methoddesc}


\begin{methoddesc}[System]{CreateSymbol}{Path, Slot, \optional{Refresh}, \optional{, Modus}\optional{, Owner}\optional{, Group}}
This method creates a new symbol on path \var{Path} connected to the slot \var{Slot}.
Optional you can set the refresh-rate, owner, group and access rights of the new symbol.

The argument \var{Path} specifies the full path to the symbol that should be 
created. \textbf{Note}: All folder on this path must exist! 

The argument \var{Slot} specifies the full qualified slot name of the slot you
want the symbol to be connected to. Thy typical format of a slot-name is 
\code{DEVICEALIAS::NAMESPACE::SLOT}. Please substitute the alias of the device
the symbol will be attached to with \code{DEVICEALIAS}. You can find the 
namespaces and slot/slot-ranges the device provide at the Chapter 
\emph{PPLT Modules}.

The argument \var{Refresh} specifies the time in seconds the value of the last
read will be cached and returned by future reads.

The optional argument \var{Modus} specifies the access-right of the new symbol.
This should be a string (i.e. \code{'600'}) containing an octal integer 
representing the access-rights like described above.

The optional argument \var{Owner} specifies the owner of the symbol. This
should be a string containing a existing user-name. By default this will be
the user marked as super-user.

The optional argument \var{Group} specifies the group, the symbol will be 
attached to. This should be a string containing the name of an existing 
group. By default this will be the group of the super-user.
\end{methoddesc}


\begin{methoddesc}[System]{DeleteSymbol}{Path}
This method will remove a symbol from the symbol tree.

The argument \var{Path} specifies the \textbf{full} path
to the symbol you want to remove. 
\end{methoddesc}


\begin{methoddesc}[System]{MoveSymbol}{From, To}
This method moves a symbol from path \var{From} into the folder specified by 
\var{To}.

The argument \var{Specifies} the symbol you want to move and the argument 
\var{To} specifies the destination folder. The destination folder have to exist.
\end{methoddesc}


\begin{methoddesc}[System]{ListSymbols}{Path}
This method will list all symbols of the folder specified by \var{Path}.

The argument \var{Path} specifies the folder.

This method will return a list of strings, even if the folder doesn't contain 
any symbols (in this case the method will return an empty list).
\end{methoddesc}


\begin{methoddesc}[System]{GetValue}{Path}
This method returns the value of the symbol specified by the argument 
\var{Path}. 

This method will return any value or a list of values on success.
\end{methoddesc}


\begin{methoddesc}[System]{SetValue}{Path, Value}
This method will set the value of the symbol specified by \var{Path} to 
\var{Value}.

The argument \var{Path} specifies the symbol, you want to set. The argument 
\var{Value} specifies the value (or even the list of values) you want to set 
to the symbol.
\end{methoddesc}


\begin{methoddesc}[System]{GetSymbolSlot}{Path}
This method returns the slot-name, a symbol is connected to. 

The argument \var{Path} specifies the \textbf{full} path
to the symbol.

This method returns a string or \code{None} on error.
\end{methoddesc}


\begin{methoddesc}[System]{GetSymbolType}{Path}
This method returns the type-name of a symbol. The argument \var{Path} 
specifies the \textbf{full} path to the symbol.
\end{methoddesc}


\begin{methoddesc}[System]{GetSymbolTimeStamp}{Path}
This method returns the timestamp of the value of a given symbol. The argument
\var{Path} specifies the full path to the symbol. \textbf{Note}: This method
will return the timestamp of the last update of the symbol-value. This is not 
always the timestamp of the last \emph{successful} update. This method will 
return a float (seconds since epoch) on success.
\end{methoddesc}


\begin{methoddesc}[System]{GetOwner}{Path}
This method will return the owner-name of a symbol. The argument
\var{Path} specifies the \textbf{full} path to the symbol. This method will
return a string on success.
\end{methoddesc}


\begin{methoddesc}[System]{ChangeOwner}{Path, Owner}
This method sets the owner of a symbol or folder. The argument \var{Path} 
specifies the \textbf{full} path to the symbol/folder. The argument 
\var{Owner} specifies the name of the new owner. This must be an existing 
username! 
\end{methoddesc}


\begin{methoddesc}[System]{GetGroup}{Path}
This method will return the name of the group a symbol or folder is attached 
to. The argument \var{Path} specifies the \textbf{full} path of the 
symbol/folder. This method will return a string containing the group-name.
\end{methoddesc}


\begin{methoddesc}[System]{ChangeGroup}{Path, Group}
This method sets the group a symbol or folder is attached to. The argument
\var{Path} specifies the \textbf{full} path to the symbol/folder. The argument
\var{Group} specifies the group-name, the folder/path will be attached to. This
must be a existing group.
\end{methoddesc}


\begin{methoddesc}[System]{GetModus}{Path}
This method will return the access-rights of the given symbol or folder.
The argument \var{Path} specifies the \textbf{full} path to the symbol/folder.
This method will returns a string formated as described above.

To process this string by your script, I recommend you to convert this string 
into a integer and then working with bit operations on it like the following 
example thats extract the rights of owner, group and other into a tuple of 
integers.
\begin{verbatim}
import PPLT
import string

pplt = PPLT.System()
[...]
tmp = pplt.GetModus("/Path/To/Symbol");
if tmp:
    tmp = string.atoi(tmp,8)    # convert string to int on base 8.
    other = tmp & 0x7;
    group = (tmp>>3) & 0x7;
    owner = (tmp>>6) & 0x7;
\end{verbatim}    
\end{methoddesc}


\begin{methoddesc}[System]{ChangeModus}{Path, Modus}
This method will set the access rights of owner, group and other
of a given symbol or folder. The argument \var{Path} specifies the 
\textbf{full} path to the symbol/folder. The argument \var{Modus} specifies
the access-rights. This should a string containing the 3 octal numbers 
specifying the rights like described above.
\end{methoddesc}


\begin{methoddesc}[System]{ClearSymbolTree}{\optional{Path}}
This method will clear (remove all) elements from the whole symbol-tree or
optional down from the given \var{Path}.
\end{methoddesc}




\subsection{Misc. methods}
\begin{methoddesc}[System]{LoadSession}{FileName}
This method will load a complete session from the given file. Note, this file
have to be in the PSF format described in the chapter 
\emph{PSF --- PPLT Session File}. The argument \var{FileName} specifies the
filename of the session file to load.
\begin{notice} 
The current session will be lost! 
\end{notice}
\end{methoddesc}


\begin{methoddesc}[System]{SaveSession}{FileName}
This method will save the current session to the given file. The used
format is the PSF format described in the chapter 
\emph{PSF --- PPLT Session File}. The argument \var{FileName} specifies
the file to save the session to.
\end{methoddesc}


\begin{methoddesc}[System]{StopAll}{}
This method stops the whole PPLT system and reset it to the init state by calling  the 
methods \method{ClearSymbolTree}, \method{StopDevices} and
\method{StopServers}.
\end{methoddesc}


\begin{methoddesc}[System]{StopDevices}{}
This method will stop and unload all loaded devices. The method can be used to
shutdown the PPLT system.
\end{methoddesc}


\begin{methoddesc}[System]{StopServers}{}
This method will stop all loaded servers and unload them. This method can be used 
to shutdown the PPLT system.
\end{methoddesc}

