\newcommand{\CoreModDesc}[1]{\newpage\subsection{#1}\index{Core Modules!#1}}
\newcommand{\CoreMod}[1]{\code{#1}}

\chapter{Core-modules Reference}
In this chapter I will describe all core (\module{pyDCPU}) modules.
A core module implements a new feature (protocol,device,...) to the
PPLT system. This is the atomic element of the PPLT devices and servers.

A core module is always a small python script with a specific interface 
zipped into an archive. This archive have to consists of at least two
files. One named \texttt{meta.xml} containing the meta-info of the
core-module. The second file have to be named \texttt{\_\_init\_\_.py}.
This file should include or contain the python source-code of the core
module.

In the following sections I will describe all core modules available until
now. Or all that I know of.



\section{Master modules}
Master modules are modules that are used in the PPLT devices. So this
modules implement the features to access a device, like the interface,
bus protocol, device commands, ...



%\CoreModDesc{Master.Interface.SendMail}
%This module implements the sending of e-mails via a
%specified SMTP host. \note{You have to have the permission 
%to send a e-mail of this host.} 



\CoreModDesc{Master.Interface.Socket}
This modules implements the basic TCP socket support for the PPLT system. 
With this module you can connect to other hosts over any tcp/ip based network
(Internet). This is used in the PPLT device \textit{PLC.FP-WEB} to connect to 
the tunneled ToolPort of a Panasoinc PLC. This module is able to hanle 
multible connections, so you don't need to load a 
\CoreMod{Master.Interface.Socket}-module for each connection you want.

\subsubsection{Parameters}
This module needs only one parameter:
\begin{tableiii}{l|p{10cm}|l}{textrm}{Parameter}{Description}{Default value}
\lineiii{TimeOut}
        {This parameter specifies the read-timeout for the socket 
         connection(s) in seconds. This should be a string with a floating 
         point number in it like '0.5' for a timeout of an half second.}
        {'0.0'}
\end{tableiii}

\subsubsection{Addresses}
This module need a address, thats specifies the host and port of the TCP 
connection. So if you connect to this module with an address "10.1.1.1:100"
a connection to the host "10.1.1.1" at the port 100 will be opend. See 
example for more details. \notice{If the connection to the host fails
the first time, it will be retryed each time you read/write from/to the 
module until the connection can be etablished.} 

\subsubsection{Example}
\begin{figure}[ht]
    \label{fig:coremod02}
    \centering
    \includegraphics[scale=1]{coremod02.png}
    \caption{This figure shows the module-map of the socket-example.}
\end{figure}    
This example opens two connection to different hosts and ports:
\lstinputlisting{coremod02.py}



%
% Serial interface module:
%
\CoreModDesc{Master.Interface.UniSerial}
The \CoreMod{Master.Interface.UniSerial} core module implements the serial
interface for the PPLT system. This module needs the Python library 
\code{pySerial}. You can get this library from the URL 
\url{http://serial.sourceforge.net}. 

The settings of this module can be changed at runtime by symbols connected to
the addresses \emph{speed}, \emph{parity} \emph{timeout}.

\subsubsection{Parameters}
This module 4 parameters to setup. 
\begin{tableiii}{l|p{10cm}|l}{textrm}{Parameter}{Description}{Default value}
\lineiii{Port}
        {This is the number of the serial interface, the moudle should use. 0 
         means \code{ttyS0} or \code{COM1}, 1 means \code{ttyS1} etc. This
         parameter is not optional!}
        {---}
\lineiii{Speed}        
        {This is the speed in Baud the serial interface will be set to. This
         parameter is not optional!} 
        {---}
\lineiii{Parity}
        {This parameter specifies the parity check (settings) of the serial 
         inteface. 'Odd', 'Even', 'None' are valid values. This parameter
         is optional.}
        {'None'}
\lineiii{TimeOut}
        {This parameter sets the timeout for reading from the serial 
         interface in seconds. All floatingpoint numbers are valid an
         \code{None} means blocking-read.}
        {\code{None}}
\end{tableiii}
\notice{Some parameters can be reset at runtime by symbols connected to this 
module at special addresses. See next paragraph for details.}

\subsubsection{Addresses}
\begin{tableiii}{l|l|p{10cm}}{testrm}{Address}{Type}{Description}
\lineiii{---}
        {\code{TStream}}
        {This is the \emph{data-channel} to the serial-interface. Meaning 
         reading from this will read from the serial interface and writeing
         into a connection to this will write into the serial-interface.}
\lineiii{speed}
        {\code{TInteger}}
        {By this connector you'll be able to change the settings of the serial
         interface at runtime. Write an integer to a connection to this address
         will (try) to reset the baudrate to the value. See example for details.}
\lineiii{parity}
        {\code{TString}}
        {Writing strings like 'even', 'odd' or 'none' into this connector, will
         reset the setting of the serial interface. Reading from the connector
         will infrom you about the current setting.}
\lineiii{timeout}
        {\code{TFloat}}
        {Writing to this connector will reset the timeout for reading from 
         the serial interface.}
\end{tableiii}

\subsubsection{Example}
\begin{figure}[ht]
    \label{fig:coremod01}
    \centering
    \includegraphics{coremod01.png}
    \caption{This figure shows the module-map of the UniSerial-example.}
\end{figure}    
This example shows how to setup the \CoreMod{UniSerial} and how to change the 
settings at runtime:
\lstinputlisting{coremod01.py}








%\CoreModDesc{Master.Interface.WGet}




%
% Transport layer of the Mewtocol:
%
\CoreModDesc{Master.Transport.MEWCOM-TL}
This module implements the transport-layer of the MEWTOCOL. The MEWTOCOL 
normally doesn't distinguish between the tansport and command layer. So
this module is only usfull in combination with the \CoreMod{MEWCOM-CL}
core-module that implements the command messages of the MEWTOCOL.

\subsubsection{Parameters}
This module needs only one parameter that specifies if a cecksum should
be calculated and adde to the MEWTOCOL message. By defualt this parameter
is set to \code{"True"}

\begin{tableiii}{l|p{10cm}|l}{textrm}{Parameter}{Description}{Default}
\lineiii{BCC}
        {This parameter specifies if a bcc (checksum) should be claculated
         for each packet send. This parameter doesn't contol the if the
         checksum of a recived packet will be checked. This will allways be 
         done (if the packet contains a checksum)\footnote{The MEWTOCOL 
         protocol allows that a packet doesn't contain a checksum. In this 
         case the checksum 'XX' will be set to indicate that the checksum
         should not be checked.}.  
        }{'True'}
\end{tableiii}       

\subsubsection{Addresses}
Each module or symbol have to set a address for the connection to this 
module. This address specifies the address of the destination the 
MEWTOCOL messages are send to. 
\begin{tableiii}{l|l|p{10cm}}{textrm}{Address}{Type}{Description}
\lineiii{0..255}
        {\code{TSequence}}
        {This address specifies the destination of the messages send.}
\end{tableiii}

\subsubsection{Example}
Because of this module is not usefull alone, please look at the example for 
the \CoreMod{MEWCOM-CL} core module.



%
% PPI module:
%
\CoreModDesc{Master.Transport.PPI}
This Core-Module implements the PPI protocol. It can be used to access the Siemens
SIMATIC S7 series PLCs. In this state of developing this module can't assable an
dispatch fragmentated packages. But in the most cases this is not needed. 

\subsubsection{Parameters}
This module needs only one parameter:
\begin{tableiii}{l|p{10cm}|l}{textrm}{Parameter}{Description}{Default}
\lineiii{Address}
        {This parameter specifies the PPI address of the PC. This is in the 
         most cases 0. But you have to set a value for this parameter!}
        {}
\end{tableiii}

\subsubsection{Addresses}
If you connect other modules or symbols to this module you'll need to specify
an address. This address is a number between 0 and 255. 
\begin{tableiii}{l|l|p{10cm}}{textrm}{Address}{Type}{Description}
\lineiii{0-255}{\code{Sequence}}
        {This address specifies the address of the device in the PPI BUS. 
         Writeing into the connection will cause a packet send to the device
         addressed by this address. Also reading from the connection will only
         return the content addressd to the setted Parameter \var{Address} and
         from the device addressd by this address.}
\end{tableiii}

\subsubsection{Example}
\begin{figure}[ht]
    \label{fig:coremod07}
    \centering
    \includegraphics{coremod07.png}
    \caption{This figure shows the module-map of the PPI-example.}
\end{figure}    
This example shows how to tunnel the PPI bus by the symbol tree. For an more
usefull example please look the the Example for the \CoreMod{S7} core-module.
\lstinputlisting{coremod07.py}



%
% ReadLine module.
%
\CoreModDesc{Master.Transport.ReadLine}
This module can be used if you want to do some line-orientated communication.

This is often used for simple protocols for the serial interface. For example
the AT commands for modems but also the Mewtocol by Panasonic (FP0 and FP1) 
uses a line-orientated protocol to send or recive commands to/from the PLC.  

So this module have to be used as a child of a module that provides
\code{TStream} or \code{TSequence} connections.

\subsubsection{Parameters}
This module needs a parameter to specify the \emph{line-end} character(s).
\begin{tableiii}{l|p{10cm}|l}{textrm}{Paramter}{Description}{Default}
\lineiii{LineEnd}
        {A hex-encoded string that the module will use as a sign for the 
         line-end. And also all data send over this module will be extended
         with this string.}
        {0A0D} 
\end{tableiii}

\subsubsection{Addresses}
This module knows only one address: no address.
\begin{tableiii}{l|l|p{10cm}}{textrm}{Address}{Type}{Description}
\lineiii{---}
        {\code{TSequence}}
        {All data written into this connector will be extended by the set 
         line-end. And also if you read from this connector it will block 
         until the set line-end character(s) appear in the data-stream.
         \textbf{Note:} The whole line will be then returned!}
\end{tableiii}

\subsubsection{Example}
\begin{figure}[ht]
    \label{fig:coremod10}
    \centering
    \includegraphics{coremod10.png}
    \caption{This figure shows the module-map of the ReadLine-example.}
\end{figure}    
This example shows on the one hand the usage of the \CoreMod{ReadLine} module.
On the other hand it is a quite good test for the \CoreMod{Echo} module. 
This example write a string into \CoreMod{ReadLine} that extends the string 
with the line-end. Then it will send this message to the \CoreMod{Echo} module.
Then it trys to read from \CoreMod{ReadLine} that trys to read a complete 
line from his parent (\CoreMod{Echo}) strips the line-end characters and
return the string. If all goes well the same string will be returned. To see
what's going on a \CoreMod{HexDump} module will be plugged between the 
\CoreMod{Echo} and \CoreMod{ReadLine} modules.
\lstinputlisting{coremod10.py}



%
% Agilent 5462x series oscilloscopes 
%
\CoreModDesc{Master.Device.5462X}



%
% Panasonic A200 Imagechecker
%
%\CoreModDesc{Master.Device.A200}



%
% GSM compatible mobile-phone.
%
\CoreModDesc{Master.Device.GSM}



%
% Command-messages for the Panasonic FP0 and FP1
%
\CoreModDesc{Master.Device.MEWCOM-CL}
This core-module implements the command messages of the MEWTOCOL. This 
protocol is used to access the Panasonic FP0 FP2 PLCs. So you can access
the markers of these PLCs using the \CoreMod{Master.Transport.MEWCOM-TL},
\CoreMod{Master.Device.MEWCOM-CL} core-modules and a interface module like 
\CoreMod{Master.Interface.UniSerial} or \CoreMod{Master.Interface.Socket}.

\subsubsection{Parameters}
This module needs no parameters to be set up.

\subsubsection{Addresses}
You have to set a address to connect a symbol to this module! With this 
address you specify the marker you want to get/set. Additional there is a
address called \code{"STATUS"} that controls the state of the PLC 
(\emph{run}/\emph{stop}).
\begin{tableiii}{l|p{2cm}|p{10cm}}{textrm}{Address}{Type}{Description}
\lineiii{STATUS}
        {\code{TBool}}
        {By this address you can control the state of the PLC. If it is 
         \code{True} then the PLC is in the \emph{Run} mode. You can also
         get the state of the PLC by reading this value.}
\lineiii{\emph{Marker}}
        {\code{TBool} or
         \code{TInteger}}
        {By setting a marker-address as the address for the connection to this 
         module, you are able to access (read/write) this marker. So the type 
         of the connection you'll get depends on the marker-address you set.} 
\end{tableiii}

\subsubsection{Example}
There are two example to show how to access a FPx PLC. The first shows how to 
setup if you want to use the \emph{ToolPort} (serial interface) and the second
shows how to setup if you want to access the PLC by the tunneled 
\emph{ToolPort} of a FP-WEB server.

\lstinputlisting{coremod11.py}


%
%
%
\CoreModDesc{Master.Device.S7}
This module implements the command-messages of the Siemens SIMATIC S7(-200). 
With this module and the \CoreMod{Master.Transport.PPI} and 
\CoreMod{Master.Interface.UniSerial} it is possible to access the S7 over the 
PPI bus. To do this you must configure the serial interface in a special way. 
Look at the example to find out how. This module can generate messages that 
cause the S7 to send the value of the requested marker. To specify the marker 
you want to get you have to connect a symbol with this module with the name of
the marker as address. It could be possible that not all availabe marker can be
read. If you know some of them please let me know.

\subsubsection{Parameters}
This module needs no parameters to be set up.

\subsubsection{Addresses}
The address you have to specify to connect a symbol to this module should be 
the name of the marker you want to get/set.
\begin{tableiii}{l|p{2cm}|p{10cm}}{textrm}{Address}{Type}{Description}
\lineiii{\emph{Marker}}
        {\code{TBool}, \code{TInteger}}
        {With the address you specify the marker you'll access. So the type 
         of the values you'll get depends on the marker you'll set. So if you 
         access an boolean marker you'll get values typed \code{TBool} if you
         access byte,word and double word Markers you'll get values types
         \code{TInteger}.}
\end{tableiii}

\subsubsection{Example}
\begin{figure}[ht]
    \label{fig:coremod08}
    \centering
    \includegraphics{coremod08.png}
    \caption{This figure shows the module-map of the S7-example.}
\end{figure}    
In this example I'll access (read/write) the markers SM0.5 (special marker 0 bit 5) and AB0
(output byte 0).

\lstinputlisting{coremod08.py}



%
% ECHO module
%
\CoreModDesc{Master.Debug.Echo}
This module echoes all data writte to it at reading from it. It can be used to
debug other modules. This is a so called \emph{root}-module. It can't be 
loaded as a child of an other module!

\subsubsection{Parameters}
This module needs no parameters!

\subsubsection{Addresses}
This module has only one addresses: no address.
\begin{tableiii}{l|l|p{10cm}}{textrm}{Address}{Type}{Description}
\lineiii{---}
        {\code{TStream}}
        {If you write into a symbol connected to this module, the data will be
         buffered and the next time you read from it the buffer (or a part of 
         it) will be returned.}
\end{tableiii}

\subsubsection{Example}
\begin{figure}[ht]
    \label{coremod06}
    \centering
    \includegraphics{coremod06.png}
    \caption{This figure shows the module-map of the Echo-example.}
\end{figure}    
This is a simple echo-example... ... the classic.

\lstinputlisting{coremod06.py}



%
% HexDump Module 
%
\CoreModDesc{Master.Debug.HexDump}
This module is a simple trafic dumper. It dumps all data going throught it
into the logging-system with loglevel \emph{debug}. So you can read it from
your logfile or \code{stderr}. Simply plug this module between two modules
to record all trafic.

\subsubsection{Parameters}
This module needs no parameters!

\subsubsection{Addresses}
This module has only one address: no address! 
\begin{tableiii}{l|p{2cm}|p{10cm}}{textrm}{Address}{Type}{Description}
\lineiii{---}
        {\code{TStream} or \code{TSequence}}
        {All data written to this module will be written to the parent and 
         read vica versa. So the type of the connection will depend on the
         type of the parent-connection.}
\end{tableiii}

\subsubsection{Example}
\begin{figure}[ht]
    \label{fig:coremod05.py}
    \centering
    \includegraphics{coremod05.png}
    \caption{This figure shows the module-map of the HexDump-example.}
\end{figure}    
This module uses the echo module to show how the trafic in read and write 
direction will be dumped:
\lstinputlisting{coremod05.py}



%
% Null module (the black hole)
%
\CoreModDesc{Master.Debug.Null}
This is a very simple module, that works like the /dev/null device file on 
Linux. If you read from this module, you'll get a string with zero-bits and
if you write into this module nothing will happens. This module swallow all
you write into it.

\subsubsection{Parameters}
This module needs no parameters to be set up.

\subsubsection{Addresses}
This module has only one valid address: no address.
\begin{tableiii}{l|l|p{10cm}}{textrm}{Address}{Type}{Description}
\lineiii{---}
        {\code{TStream}}
        {This will be the connection to the \CoreMod{Null} module. Id you
         write into this connection the module will swallow all data and
         nothing will happen. If you read from this module you'll get a 
         stream of zero-bits.}
\end{tableiii}

\subsubsection{Example}
\begin{figure}[ht]
    \label{fig:coremod09}
    \centering
    \includegraphics{coremod09.png}
    \caption{This figure shows the module-map of the Null-Example.}
\end{figure}    
This example shows a configuration to debug a non \emph{root} module. In this 
case it is the \CoreMod{Master.Transport.PPI} core-module. To get some 
information about the packages the PPI module sends the core-module 
\CoreMod{Master.Debug.HexDump} is used.  \notice{This example raises an 
exception, because the PPI module expect a correct answer from the 
\emph{virtual} device (in this case the core-module 
\CoreMod{Master.Debug.Null}).
\lstinputlisting{coremod09.py}



%
% Random-generator module
%
\CoreModDesc{Master.Debug.Random}
This module implements a random-generator, that can generate data for every 
symbol-type supported by the \module{pyDCPU}. So this module can be used to 
test almost all facilities of the system. 

\subsubsection{Parameters}
This module needs no parameters. 

\subsubsection{Addresses}
\begin{tableiii}{l|l|p{10cm}}{testrm}{Address}{Type}{Description}
\lineiii{Bool}
        {\code{TBool}}
        {Provide a random boolean value.}
\lineiii{Integer}
        {\code{TInteger}}
        {Provide a random integer value between 0 and 100.}
\lineiii{Float}
        {\code{TFloat}}
        {Provide a random floating point number between 0 and 1.}
\lineiii{String}
        {\code{TString}}
        {Provide a random string of a random length between 1 and 79 containg 
         printable characters.}
\lineiii{ArrayBool}
        {\code{TArrayOfBool}}
        {The name tells everything. The length of the array is randomly 
         between 1 and 3.}
\lineiii{ArrayInteger}
        {\code{TArrayOfInteger}}
        {Array of integer of random length between 1 and 3.}
\lineiii{ArrayFloat}
        {\code{TArrayOfFloat}}
        {Array of floating point numbers with variable length between 1 and 
         3.}
\lineiii{ArrayString}
        {\code{TArrayOfString}}
        {Array of string with variable length (of array) between 1 and 3.}
\lineiii{Stream}
        {\code{TStream}}
        {Provides an random data string. The data are printable character.
         The number of bytes returned is less than equeal the number you 
         wanted to read. To read from a symbol connected to this address,
         please use the method \method{SymbolTreeRead().}}
\lineiii{Sequence}
        {\code{TSequence}}
        {A sequence of random data. The data contains only printable 
         characters and the length varies between 1 and 79.}
\end{tableiii}

\subsubsection{Example}
This is a simple example that uses all types:
\lstinputlisting{coremod03.py}



%
% Statistic module
%
\CoreModDesc{Master.Debug.Statistic}
This module collects statistical values of the data flowing through it. You 
can use this module to improve or debug your own modules or even to observe
the PPLT system. This module is quite easy to use: simply plug this between 
two other modules and you are able to observe the trafic between them.
\notice{You can only observe modules that provides a \code{TStream} or 
\code {TSequence} connection.}

\subsubsection{Parameters}
This module needs no parameters!

\subsubsection{Addresses}
There are several addresses provideing the statistical data. If you use no 
address to connect to this module a \code{TStream} or \code{TSequence} 
connection to the parent will be returned. This connection to the parent will
be used to collect the statistical data. 

\begin{tableiii}{l|p{2cm}|p{10cm}}{textrm}{Address}{Type}{Description}
\lineiii{---}
        {\code{TStream} or \code{TSequence}}
        {This is the data-tunnel to the parent of the 
         \CoreMod{Statistic}-module. The type of the connection depends on
         the type of the connection to the parent module. You see, it is only
         possible to connect the \CoreMod{Statistic}-module to a parent that
         provides a \code{TStream} or \code{TSequence} connection.}
\lineiii{read\_data}
        {\code{TInteger}}
        {Returns the number of bytes read from the parent.}
\lineiii{write\_data}
        {\code{TInteger}}
        {Returns the number of bytes written to the parent.}
\lineiii{read\_speed}
        {\code{TFloat}}
        {Returns the number of bytes per second read from parent since 
         module-loading. This value is the average of the whole time since
         module-loading.}
\lineiii{write\_speed}
        {\code{TFloat}}
        {Returns the number of bytes written to the parent since the module 
        was loaded. This value is the average of the whole time.}
\lineiii{error}
        {\code{TInteger}}
        {Returns the number of read/write-errors since the module was loaded.
         This must not be an indicatior for bugs in the parent modules because
         also timeouts will be recorded.}
\end{tableiii}

\subsubsection{Example}
In this example the serial interface will be exported by the symbol tree.
The \CoreMod{Statistic}-module records the trafic throught the serial 
interface.
\lstinputlisting{coremod04.py}



\newpage
\section{Export modules}
\CoreModDesc{Export.JVisu}
\CoreModDesc{Export.SimpleExport}
\CoreModDesc{Export.PPLTWeb}

