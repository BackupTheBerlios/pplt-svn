\newcommand{\PPLTModDesc}[1]{\subsection{#1}\index{PPLT Modules!#1}}
\newcommand{\PPLTMod}[1]{\code{#1}}
\newcommand{\PPLTDev}[1]{\PPLTMod{#1}}
\newcommand{\PPLTSrv}[1]{\PPLTMod{#1}}


\chapter{PPLT Module Reference}
In this chapter i will describe all available PPLT modules. A PPLT module is
server or device, that consists of one or more so called 
\textit{core modules}. The main idea of having an abstract device instead of
many (more flexible) core modules, is the easy handling of a (single) simple
device instead of dealing with a couple of modules.

Technical, a PPLT module is an XML file describing how to combine the 
core-modules to get the support for a special device or system. 

For example; If you want to access the markers of a Siemens SIMATIC S7-200 
by the PPI bus you have to load the core-module for the serial interface, the 
module for the PPI bus and at the end the module for the S7-200. Each of these
modules has a couple of parameters, to get the system running. Sometimes you 
will need to know a lot about the bus-system to know how to setup the single 
core modules. Instead of this you can load a single PPLT module called 
\PPLTMod{'PLC.S7-200'}. This module needs only 3 parameters to setup the 
support for the S7. This would be much easier.




\section{Devices}
This section describe all devices. All devices are grouped in classes. The 
name of the device consists of the full class path and the name of the 
specific device, divided by a single dot. For example: 
\PPLTMod{Debug.RandomGenerator}.






\PPLTModDesc{Debug.RandomGenerator}
The PPLT module \PPLTMod{Debug.RandomGenrator} implements a simple random 
generator, thats provide random value in different types. This is the 
simplest device. It needs no additional python libraries nor any special 
hardware to run. So it can easily be used to test the PPLT.

\subsubsection{Parameters}
This device needs no parameters to be set up. 

\subsubsection{Namespaces and slots}
This device provide only one namespace called \code{Generator}. This 
namespace contains 4 slots. Each slot provide a random value for a different
type and different range.

\paragraph{Slots of namespace \texttt{Generator}:}
Each slot returns a value of the type by his name.
\begin{tableiii}{l|l|l}{textrm}{Slotname}{Type}{Description}
\lineiii{Bool}
        {Bool}
        {Returns randomly \code{True} or \code{False}}
\lineiii{uInteger}
        {uInteger}
        {Returns a unsigned integer between 0-100.}
\lineiii{Float}
        {Float}
        {Returns a floating-point number between 0-1.}
\lineiii{Double}        
        {Double}
        {Returns a floating-point number between 0-1.}
\end{tableiii}

\subsubsection{Example}
\begin{verbatim}
import PPLT

pplt = PPLT.System();
pplt.LoadDevice("Debug.RandomGenerator", "alias", {})
\end{verbatim}

This example loads the device as \code{"alias"}. This alias should be replaced
by the alias you want to give to the loaded device-instance. You'll need this 
alias later to unload the device or to connect symbols to it.









\PPLTModDesc{Mobile.GSMMobilePhone}
This device can access a GSM compatible mobile phone by the serial interface. 
It can read out some status information  like battery level, signal quality, 
etc... In the near future this device will also learn to send SMS. 

\begin{notice}
Because this device uses the serial interface, you'll need to have the 
\module{pyserial} python library installed. Please check this before you use 
this device.
\end{notice}

\subsubsection{Parameters}
This device needs some parameters\footnote{Please note, that all 
parameter-value have to be strings.} be set up. At least you have to set 
the port (number of the serial interface) where the mobile phone is connected 
to the computer. Additional you can set the speed (in baud) of the interface.
\begin{tableiii}{l|l|l}{textrm}{Parameter}{Description}{Default value}
\lineiii{Port}
        {Sets the number of the serial interface. Note: COM1 (ttyS0) = 0, 
        COM2 = 1, ...} 
        {"0"}
\lineiii{Speed}
        {Sets the speed in baud of the serial interface.}
        {"9600"}
\end{tableiii}

\subsubsection{Namespaces and slots}
This device provides only one namespace called \code{GSM}. This namespace
contains 6 slots. These slots provides status information about the 
mobile-phone.

\paragraph{Slots of namespace \texttt{GSM}:}
There are several slots, each providing a status value of the mobile phone.
\begin{tableiii}{l|l|l}{textrm}{Slotname}{Type}{Description}
\lineiii{battery}
        {uInteger}
        {The battery level in percent.}
\lineiii{network}
        {uInteger}
        {The network-status of the mobile phone.}
\lineiii{quality}
        {uInteger}
        {The signal quality.}
\lineiii{errorrate}
        {uInteger}
        {The error-rate of the network-connection.}
\lineiii{manufacturer}
        {String}
        {Manufacturer name.}
\lineiii{model}
        {String}
        {Model-name.}
\end{tableiii}


\subsubsection{Example}
This example loads the \PPLTDev{GSMMobilePhone} module on port 0 (COM1) with 
speed 9600 baud as the alias \code{'gsm'}. Then it creates a symbol 
\code{/manu} that will be connected with the \code{manufacturer} slot of the 
device. And at the end the symbol will be read out.
\begin{verbatim}
import PPLT

pplt = PPLT.System()
pplt.LoadDevice("Mobile.GSMMobilePhone","gsm",{'Port':'0', 'Speed':'9600'})
pplt.CreateSymbol("/manu","gsm::GSM::manufacturer","String")

print pplt.GetValue("/manu")
\end{verbatim}






\PPLTModDesc{PLC.Panasonic-FPX}
This device implements the support for the Panasonic FP0 and FP2 PLCs. With 
this device you can read/write the markers of the PLC connected to the PC by 
the so called ToolPort or over the Mewtocol-BUS. 

You can also access a FP0 by the FP-WEB server, who tunnels the toolport to a 
TCP port by the \PPLTDev{PLC.Panasonic-FPWEB} device. 

\subsubsection{Parameters}
This device needs at least the number of the serial interface and
the Mewtocol-address of the PLC.
\begin{notice}
All parameter-values have to be strings!
\end{notice}

\begin{tableiii}{l|p{10cm}|l}{textrm}{Parameter}{Description}{Default value}
\lineiii{Port}
        {The number of the serial interface, where the PLC is connected to. 
        (Note: COM1(ttyS0) = 0, COM2(ttyS1) = 1,...)}
        {}
\lineiii{Address}
        {The Mewtocol-address of the PLC in the Mewtocol BUS. Note: If you use
        the ToolPort to access the PLC you can take any number between 0 and
        255. (The machine will ignore the address.)}
        {}
\end{tableiii}

\subsubsection{Namespaces and slots}
Also this device provide only one namespace called \code{Marker}. This 
namespace contains the slot \var{STATUS} and the slot-range \var{Marker}. A
slot-range is a placeholder of a couple of slots. In this context it is a
placeholder for all markers of the PLC. So if you want to access a marker,
for example \code{Y0}\footnote{The first output pin.}, you should replace
the slot-range by the name of the marker: \code{Alias::Marker::Y0}. 
You have to figure out the type of the slot by your self, if you use
a slot-range! But so far, if you access a boolean marker, use \code{'Bool'}
if you access an integer marker (byte, word, double word) please use 
the unsigned integer \code{'uInteger'} as type.
\begin{notice}
Please use uppercase for the name of the marker!
\end{notice}

\paragraph{Slots of the namespace \texttt{Marker}:}
There is only on slot in this namespace. But you can access all markers 
of the PLC like described above.
\begin{tableiii}{l|l|p{10cm}}{textrm}{Slotname}{Type}{Description}
\lineiii{STATUS}
        {Bool}
        {This slot controls the status of the PLC, if this slot is 
        \code{True} the PLC is in the \emph{Run} mode if it is
        \code{False} it is in the \emph{Stop} mode. You can also
        set the mode by writing into this slot.}
\end{tableiii}

\subsubsection{Example}
In this example I show you how to setup the \PPLTDev{PLC.Panasonic-FPX} 
device. Then I create a symbol for the status bit and one for the first 
output-bit (Y0). Then I set the PLC into the \emph{Run} mode. At the end the
script will wait for a second and then it will set the first output-bit
to \code{True}.

\begin{verbatim}
import PPLT
import time

pplt = PPLT.System()
pplt.LoadDevice("PLC.Panasonic-FPX","fp0",{"Port":"0", "Address":"1"});

pplt.CreateSymbol("/stat","fp0::Marker::STATUS","Bool");
pplt.CreateSymbol("/y0", "fp0::Marker::Y0", "Bool");

# set the PLC into run-mode:
pplt.SetValue("/stat",True);

# wait a second:
time.sleep(1);

# set Y0 to 1:
pplt.SetValue("/y0",True);
\end{verbatim}






\PPLTModDesc{PLC.FPWEB}
This device implements the access to a Panasonic FP0 or FP2 over the toolport
tunneled by the Panasonic FP-WEB server. This device is quiet equal to the
\PPLTDev{PLC.Panasonic-FPX} device. So it provides the same namespace with
the same slots and slot-ranges.

If you want to access the FP0 or FP2 directly by the toolport, please use
the \PPLTDev{PLC.Panasonic-FPX} device instead.

\subsubsection{Parameters}
This device needs only two parameters to be set up correctly. The network
address of the web-server and the Mewtocol address of the PLC.
\begin{tableiii}{l|p{10cm}|l}{textrm}{Parameter}{Description}{Default value}
\lineiii{NetAddr}
        {The network address of the web-server and the port of the tunneled
        toolport in the format \code{ADDRESS:PORT}. For example 
        \code{10.1.1.4:9094}.}
        {}
\lineiii{MewAddr}
        {The Mewtocol address of the PLC connected to the web-server. In
        the normal case, the web-server will be plugged on the toolport
        of the PLC so this number could be any between 0 and 255 because
        the PLC will ignore the destination address field.}
        {\code{1}}
\end{tableiii}

\subsubsection{Namespaces and slots}
Also this device provide only one namespace called \code{Marker}. This 
namespace contains the slot \var{STATUS} and the slot-range \var{Marker}. A
slot-range is a placeholder of a couple of slots. In this context it is a
placeholder for all markers of the PLC. So if you want to access a marker,
for example \code{Y0}\footnote{The first output pin.}, you should replace
the slot-range by the name of the marker: \code{Alias::Marker::Y0}. 
You have to figure out the type of the slot by your self, if you use
a slot-range! But so far, if you access a boolean marker, use \code{'Bool'}
if you access an integer marker (byte, word, double word) please use 
the unsigned integer \code{'uInteger'} as type.
\begin{notice}
Please use uppercase for the name of the marker!
\end{notice}

\paragraph{Slots of the namespace \texttt{Marker}:}
There is only on slot in this namespace. But you can access all markers 
of the PLC like described above.
\begin{tableiii}{l|l|p{10cm}}{textrm}{Slotname}{Type}{Description}
\lineiii{STATUS}
        {Bool}
        {This slot controls the status of the PLC, if this slot is 
        \code{True} the PLC is in the \emph{Run} mode if it is
        \code{False} it is in the \emph{Stop} mode. You can also
        set the mode by writing into this slot.}
\end{tableiii}

\subsubsection{Example}
In this example I will show you how to load the \PPLTDev{PLC.FPWEB} device.
The script will then do the same like the example script of the 
\PPLTDev{PLC.Panasonic-FPX} device. It will create 2 symbols, one for
the status bit and one for the first output bit (Y0), then it will
set the PLC into the \emph{Run} mode and will set the Y0 to True.

\begin{verbatim}
import PPLT
import time

pplt = PPLT.System()
pplt.LoadDevice("PLC.FPWEB","fp",{"NetAddr":"10.1.1.100:9094", "MewAddr":"1"});

pplt.CreateSymbol("/stat","fp::Marker::STATUS","Bool");
pplt.CreateSymbol("/y0", "fp::Marker::Y0", "Bool");

# set the PLC into run-mode:
pplt.SetValue("/stat",True);

# wait a second:
time.sleep(1);

# set Y0 to 1:
pplt.SetValue("/y0",True);
\end{verbatim}






\PPLTModDesc{PLC.S7-200}
This device implements the access to a Siemens SIMATIC 
S7-200\footnote{I've only tested it with a S7-200 maybe other also working 
fine. Please let me know if it works for you.} PLC. With this device you can 
read/write the markers of a Siemens PLC. Additional you can get some 
statistical values about the PPI BUS line number of bytes send/received.
\begin{notice}
This device implements only the access over a PPI cable!
\end{notice}


\subsubsection{Parameters}
To setup the device you need to set at least the number of the serial 
interface used and the PPI address of the PC and the PLC.
\begin{tableiii}{l|l|l}{textrm}{Parameter}{Description}{Default value}
\lineiii{Port}
        {Number of the serial interface to be used. (COM1=0, COM2=1,...)}
        {0}
\lineiii{PCAddr}
        {The PPI address of the PC. This would normally 0(Master).}
        {0}
\lineiii{S7Addr}
        {The PPI address of the PLC. Normaly a number between 0 and 32.}
        {2}
\end{tableiii}

\subsubsection{Namespaces and slots}
This device provides two namespaces. One called \code{PPIStatistic}, provides
slots for statistical values about the PPI BUS. The other one called 
\code{Marker} provides a slot-range named \code{Merkers}.

A slot-range is a placeholder for a whole range of slots. In this case it
is a placeholder for all markers of the PLC. So if you want to access a
marker, please replace the slot-range by the name of the marker you want 
to use. For example if you want to connect a symbol to the marker
\code{SMB28} please use \code{"ALIAS::Marker::SMB28"} as the 
slot-name\footnote{Please use only uppercase letters for the marker-name.}. 


\paragraph{Slots of the namespace \texttt{PPIStatistic}:}
This namespace contains some slot for statistical information about the 
underlying PPI BUS. 
\begin{tableiii}{l|l|p{10cm}}{textrm}{Slotname}{Type}{Description}
\lineiii{read\_data}
        {uInteger}
        {This slot returns the number of received bytes.}
\lineiii{write\_data}
        {uInteger}
        {This slot returns the number of send bytes.}
\lineiii{read\_speed}
        {uInteger}
        {Returns the number of bytes received per second.}
\lineiii{write\_speed}
        {uInteger}
        {Returns the number of bytes send per second.}
\lineiii{error}
        {uInteger}
        {Counts the errors at the transport-layer.}
\end{tableiii}


\paragraph{Slots of the namespace \texttt{Marker}:}
This namespace contains only on slot-range. This slot-range is a placeholder of
all markers of the PLC. So if you want to connect a symbol with a marker of 
the PLC you have to choose a slot name like 
\code{'ALIAS::Marker::MARKERNAME'}. Please replace \code{ALIAS} by the alias 
the device will have after being loaded and replace \code{MARKERNAME} by the 
marker address of the one you want the symbol being connected to.

Because the PPLT system can't know what type a specific marker has, you have 
to set the type by your self. In this case you have to choose the type 
\code{Bool} if it is a boolean value and the type \code{uInteger} if it is an 
integer value.


\subsubsection{Example}
This example shows how to use the device \PPLTDev{PLC.S7-200}. At first the 
device will be loaded. Then two folder will be created in the symbol-tree. The 
first (\code{/S7}) contains two symbols of PLC markers and the second folder
(\code{/S7/PPI}) contains a symbol holding the read data.

The value of the symbol \code{/S7/AB0} will be read and then the inverse of 
this value will be written back into the symbol. Then the symbol 
\code{/S7/SMB28} will be read. And at the end the number of received bytes 
will be read out of the symbol \code{/S7/PPI/read}.
\begin{verbatim}
import PPLT

pplt = PPLT.System()

pplt.LoadDevice("PLC.S7-200", "s7", {"Port":"0", "PCAddr":"0", "S7Addr":"2"});

pplt.CreateFolder("/S7");
pplt.CreateFolder("/S7/PPI");

pplt.CreateSymbol("/S7/AB0", "s7::Marker::AB0", "uInteger");
pplt.CreateSymbol("/S7/SMB28", "s7::Marker::SMB28", "uInteger");
pplt.CreateSymbol("/S7/PPI/read", "s7::PPIStatistic::read_data", "uInteger");

val = pplt.GetValue("/S7/AB0");
print val;
val = val ^ 0xff  #inverse
pplt.SetValue("/S7/AB0",val);

print pplt.GetValue("/S7/SMB28");

print pplt.GetValue("/S7/PPI/read");
\end{verbatim}




\PPLTModDesc{Measure.AGILENT-5462X}
This device implements the access to a Agilent oscilloscope of the 5462X 
series. With this device you can control the oscilloscope, for example you can
measure the frequency of a signal. This device supports only the serial
interface to the Agilent, GPIB or something like that is not supported.

I have written this device more or less as a prove of concept. There are
much more possibilities for measurements, but i had'n implemented them.
So if you have some experiences in programming Python and access to such
a device please contact me.

\subsubsection{Parameters}
This device needs only few parameters, but you may have to do some settings 
at your oscilloscope. 

\begin{notice} You may have to do some settings on your oscilloscope.
At first set the interface to \code{serial}, then disable \code{parity},
set the flow-control to \code{RTS/TSR} and set the speed to \code{57600} baud.
\end{notice}

Following parameters are needed to setup the device.
\begin{tableiii}{l|p{10cm}|l}{textrm}{Parameter}{Description}{Default value}
\lineiii{Port}
        {This is the number of the serial interface. (COM1=0, COM2=1, ...)}
        {0}
\lineiii{Primary}
        {This is the primary signal source to be used by the oscilloscope.
        \code{A1} means the first analog input, \code{A2} means the second, 
        ...}
        {A1}
\lineiii{Secondary}
        {This is the secondary signal source. Needed, if you want to compare two signals.}
        {A2}
\end{tableiii}

\subsubsection{Namespaces and slots}
This device provides only one namespace with several slots in it. This namespace is called
\code{Values}. Each slot starts a specific measurement if someone reads out of it.

\paragraph{Slots of the namespace \texttt{Values}:} 
\begin{tableiii}{l|l|l}{textrm}{Slot}{Type}{Description}
\lineiii{amp}
        {Double}
        {The amplitude of the signal at the primary input.}
\lineiii{freq}
        {Double}
        {The frequency of the signal at the primary input}
\lineiii{phase}
        {Double}
        {Phasediff between the signals at the primary and secondary input.}
\lineiii{max}
        {Double}
        {Maximum of the signal at the primary input.}
\lineiii{min}
        {Double}
        {Minimum of the signal at the primary input.}
\lineiii{pp}
        {Double}
        {Peek-Peek value of the signal at the primary input.}
\lineiii{width}
        {Double}
        {Pulse width of the signal at the primary input.}
\end{tableiii}        


\subsubsection{Example}
In this example I will show you how to setup the device and make some measurements.
\begin{verbatim}
import PPLT

pplt = PPLT.System()

# COM1, Primary=Secondary=Analog1
pplt.LoadDevice("Measure.AGILENT-5462X","agi",
                {'Port':'0', 'Primary':'A1', 'Secondary':'A1'})


#symbols:
pplt.CreateSymbol("/amp","agi::Values::apm","Double");
pplt.CreateSymbol("/freq","agi::Values::freq","Double");
pplt.CreateSymbol("/phase","agi::Values::phase","Double");

print pplt.GetValue("/amp");
print pplt.GetValue("/freq");
print pplt.GetValue("/phase"); #should be zero.

\end{verbatim}







\section{Servers}
In this section I'll describe all available servers for the PPLT system. 
Server are responsible to export the symbol-tree to other system like 
visualizations or what ever. 

Like devices servers are also grouped by classes. A full qualified server-name
consists of the whole class-path a the name divided by a single dot. For 
example: \PPLTSrv{Web.PPLTWebServer}.

All servers running in there own thread, so they can work while the 
main-application blocks.


\PPLTModDesc{Web.PPLTWebServer}
This server exports the symbol-tree as a web-server so you can browse the 
symbol-tree with your favorite FireFox. This server supports a basic 
authentication so the \code{DefaultUser} attribute will be ignored.

\subsubsection{Parameters}
To setup the server you have to set at least the address and the port, the 
server will listen for new connections.
\begin{tableiii}{l|l|l}{textrm}{Parameter}{Description}{Default value}
\lineiii{Address}
        {The address the server will listen for new connections.}
        {127.0.0.1}
\lineiii{Port}
        {The port the server will be listen on.}
        {8080}
\end{tableiii}

\subsubsection{Example}
This example needs no special hard- nor software. It loads the random-generator
creates some symbols and starts the web-server. Now you can browse through the
symbol tree by going to the URL \code{http://127.0.0.1:8080}.
\begin{verbatim}
import time
import PPLT

pplt = PPLT.System()

# Load random
pplt.LoadDevice("Debug.RandomGenerator", "rand", {})

# create symbols
pplt.CreateFolder("/rand")
pplt.CreateSymbol("/rand/bool", "rand::Generator::Bool", "Bool")
pplt.CreateSymbol("/rand/int", "rand::Generator::uInteger", "uInteger")
pplt.CreateSymbol("/rand/float", "rand::Generator::Double", "Double")

# load server
pplt.LoadServer("Web.PPLTWebServer", "web", "admin", 
                {"Address":"127.0.0.1", "Port":"8080"})

# do nothing loop:
while 1: time.sleep(1)
    
\end{verbatim}





\PPLTModDesc{Visu.JVisuServer}
This server exports the symbol-tree for the Java visualization JVisu. 
(\url{http://jvisu.sourceforge.net}). The protocol used by \code{JVisuSocket} 
doesn't know any authentication so you need to set the \var{DefaultUser}
\textbf{carefully}.

\subsubsection{Parameters}
You need to set at least the address and the port the server will listen on 
for new connections.

\begin{tableiii}{l|l|l}{textrm}{Parameter}{Description}{Default value}
\lineiii{Address}
        {The address the server will listen on for new connections.}
        {127.0.0.1}
\lineiii{Port}
        {The port the server will listen on.}
        {2200}
\end{tableiii}        

\subsubsection{Example}
This example will do the same like the example of the 
\PPLTSrv{Web.PPLTWebServer}. but in this case it will start a 
JVisuSocketServer instead of a web-server.
\begin{verbatim}
import time
import PPLT

pplt = PPLT.System()

# Load random
pplt.LoadDevice("Debug.RandomGenerator", "rand", {})

# create symbols
pplt.CreateFolder("/rand")
pplt.CreateSymbol("/rand/bool", "rand::Generator::Bool", "Bool")
pplt.CreateSymbol("/rand/int", "rand::Generator::uInteger", "uInteger")
pplt.CreateSymbol("/rand/float", "rand::Generator::Double", "Double")

# load server
pplt.LoadServer("Visu.JVisuServer", "jv", "admin", 
                {"Address":"127.0.0.1", "Port":"2200"})

# do nothing loop:
while 1: time.sleep(1)
    
\end{verbatim}


\PPLTModDesc{RPC.SimpleExport}
\PPLTMod{RPC.SimpleExport} is an XML-RPC server, thats exports some functions 
to access the symbol-tree. So you can access the symbols by nearly any 
programming-language on any system. 

\subsubsection{Parameters}
You need to set at least the address and the port the server will listen on 
for new connections. 
\begin{tableiii}{l|l|l}{textrm}{Parameter}{Description}{Default value}
\lineiii{Address}
        {The address, the server will listen on for new connections.}
        {127.0.0.1}
\lineiii{Port}
        {The port, the server will listen on.}
        {4711}
\end{tableiii}        

\subsubsection{Functions}
The \PPLTSrv{RPC.SimpleExport}-sever is an XML-RPC server, that exports 
functions you can call from remote side. This section lists all available 
functions and what attributes are needed. 

\begin{funcdescni}{logon}{UserName, Passwd}
This function returns a session ID you can use to authenticate yourself. This 
ID will be used as an additional attribute for the \function{set}, 
\function{get} \function{listsymbols}, \function{listfolders} and 
\function{logoff} function calls.

The attribute \var{UserName} specifies the name of the user. And
the attribute \var{Passwd} specifies the password.
\end{funcdescni}


\begin{funcdescni}{logoff}{SessionID}
This function closes a session opened by \function{logon}.
The attribute \var{SessionID} is the id returned by the
\function{logon}-function-call.
\end{funcdescni}


\begin{funcdescni}{get}{SymbolPath,\optional{SessionID}}
This function will return the value of the symbol pointed by \var{SymbolPath}.

The attribute \var{SymbolPath} specifies the full path of the symbol you want 
to read.
The optional attribute \var{SessionID} specifies the session you may opened by 
a \function{logon}-function-call. If you missed the \var{SessionID}, the 
rights of the default user are used to access the symbol. 

The function returns \code{None} on error.
\end{funcdescni}


\begin{funcdescni}{set}{SymbolPath, Value, \optional{SessionID}}
This function will set the value of the symbol pointed by \var{SymbolPath} to 
\var{Value}.

The attribute \var{SymbolPath} specifies the full path of the symbol you want 
to read.

The optional attribute \var{SessionID} specifies the session you may opened by 
a \function{logon}-functioncall. If you missed the \var{SessionID}, the 
rights of the default user are used to access the symbol. 

The function returns \code{True} on success and \var{False} otherwise.
\end{funcdescni}


\begin{funcdescni}{listfolders}{Path, \optional{SessionID}}
This function will list all folders at \var{Path}.
\end{funcdescni}


\begin{funcdescni}{listsymbols}{Path, \optional{SessionID}}
This function will list all symbols at \var{Path}.
\end{funcdescni}


\subsubsection{Example}
This example consists of two parts. The first is the server showing
how to setup the server-module. The second part is a small script, that
access the server and reads some values.

This is the server, it does nearly the same like the other server examples do.
\begin{verbatim}
import time
import PPLT

pplt = PPLT.System()

# Load random
pplt.LoadDevice("Debug.RandomGenerator", "rand", {})

# create symbols
pplt.CreateFolder("/rand")
pplt.CreateSymbol("/rand/bool", "rand::Generator::Bool", "Bool")
pplt.CreateSymbol("/rand/int", "rand::Generator::uInteger", "uInteger")
pplt.CreateSymbol("/rand/float", "rand::Generator::Double", "Double")

# load server
pplt.LoadServer("RPC.SimpleExport", "sx", "admin", 
                {"Address":"127.0.0.1", "Port":"4711"})

# do nothing loop:
while 1: time.sleep(1)
    
\end{verbatim}


This is the client script:
\begin{verbatim}
import xmlrpclib

srv = xmlrpclib.ServerProxy("http://127.0.0.1:4711")

#logon:
session = srv.logon("user","pass")     # YOU HAVE TO SET HERE REAL USER/PASS

#list folders in /
print srv.listfolders("/", session)

#list symbols in /rand
print srv.listsymbols("/rand", session)

#value of /rand/bool
print srv.get("/rand/bool", session)

#produce an error (/rand/bool is read-only!):
print srv.set("/rand/bool", True, session)
\end{verbatim}
